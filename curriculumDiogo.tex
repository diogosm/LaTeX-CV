\documentclass[11pt,a4paper,sans]{moderncv}

%% ModernCV themes
\moderncvstyle{casual}
\moderncvcolor{blue}
\renewcommand{\familydefault}{\sfdefault}
\nopagenumbers{}

%% Character encoding
\usepackage[utf8]{inputenc}

%% Adjust the page margins
\usepackage[scale=0.75]{geometry}

%% Personal data
\firstname{Diogo}
\familyname{Soares}
\title{Curriculum Vitae}
\address{Rua L, Conjunto Ribeiro Jr. Cidade Nova 1. Manaus, AM.}{}
\mobile{+55~(092)~9177~1373}
%\phone{+2~(345)~678~901}
%\fax{+3~(456)~789~012}
\email{soares.diogom@gmail.com}
\homepage{diogosm.github.io}
%\extrainfo{additional information}
\photo[64pt][0.4pt]{profile}
%\quote{Some quote (optional)}

%%------------------------------------------------------------------------------
%% Content
%%------------------------------------------------------------------------------
\begin{document}
\makecvtitle

\section{Dados Pessoais}
\cvlistitem{\textbf{Nome Completo:} Diogo Soares Moreira}
\cvlistdoubleitem{\textbf{Idade:} 27}{\textbf{Estado Civil:} Casado}
\cvlistitem{\textbf{Endereço: }Rua L, Conjunto Ribeiro Jr. Cidade Nova 1. Manaus, AM.}
\cvlistitem{\textbf{Celular:} 91771373}
\cvlistitem{\textbf{Email:} soares.diogom@gmail.com}

\section{Formação}
\cventry{2018--}{Doutorado em andamento em Informática}{Instituto de Computação}{}{}{Universidade Federal do Amazonas.}
\cventry{2017--2018}{Mestrado em Informática}{Instituto de Computação}{}{}{Universidade Federal do Amazonas.}
\cventry{2008--2011}{Bacharelado em Ciência da Computação}{Instituto de Computação}{}{}{Universidade Federal do Amazonas.}  % arguments 3 to 6 can be left empty

\section{Formação Complementar}
\cventry{2015--2015}{Gestão da Segurança da Informação - NBR 27001 e NBR 27002}{}{}{}{Escola Superior de Redes (ESR). 40h.}
\cventry{2012--2012}{Escola de Verão da Maratona de Programação}{Instituto de Computação}{}{}{Universidade Estadual de Campinas (UNICAMP). 160h.}
\cventry{2012--2012}{Curso de Atualização em Redes de Computadores}{Instituto de Computação}{}{}{Universidade Federal do Amazonas (UFAM). 40h.}  % arguments 3 to 6 can be left empty
\cventry{2011--2011}{Programação para Web - PHP e JQuery}{Instituto de Computação}{}{}{Universidade Federal do Amazonas (UFAM). 30h.}  % arguments 3 to 6 can be left empty

\section{Atuação Profissional}
\cventry{2014--}{Servidor Público. Cargo: Analista de Tecnologia da Informação}{}{Carga horária: 40}{Centro de Tecnologia da Informação e Comunicação (CTIC)}{Universidade Federal do Amazonas (UFAM).}
\cventry{2017--2017}{Professor de Ensino Técnico}{Disciplina(s): Segurança da Informação}{Carga horária: 30}{}{Centro de Educação Tecnológica do Amazonas (CETAM).}
\cventry{2014--2014}{Professor de pós-graduação em nível EaD}{Disciplina(s): Introdução à Educação a Distância e suas ferramentas}{Carga horária: 28}{}{Instituto Federal do Amazonas (IFAM).}
\cventry{2008--2011}{Bolsista do programa ITI. Projeto: Sistema de Monitoramento e Armazenamento de Dados de FaUna Terrestre e MIcro-clima Gerados por Sensores Móveis e Fixos (SAUIM)}{}{}{Instituto de Computação}{Universidade Federal do Amazonas (UFAM).}

\section{Idiomas}
\cvitemwithcomment{Inglês}{Compreende Bem, Fala Razoavelmente, Lê Bem, Escreve Bem.}{}

\section{Produção Técnica}
\cvitem{Artigo Publicado}{Internet of Things (IoT): technological indicators from patent analysis. In: International Joint Conference on Industrial Engineering and Operations Management, Lisboa. IJCIEOM 2018.}
\cvitem{Artigo Publicado}{A Feasibility Study of Watchdogs on Opportunistic Mobile Networks. In: IEEE Symposium on Computers and Communications, Natal. ISCC 2018.}
\cvitem{Artigo Publicado}{Avaliação Experimental da Eficácia de um Watchdog em Redes Oportunistas Móveis. In: Simpósio Brasileiro de Redes de Computadores e Sistemas Distribuídos, Campos do Jordão. SBRC 2018.}
\cvitem{Artigo Publicado}{Gerenciamento de Buffer Baseado em Egoísmo para Redes Tolerantes a Atrasos e Desconexões. In: Simpósio Brasileiro de Redes de Computadores e Sistemas Distribuídos, Belém. SBRC 2017.}
\cvitem{Artigo Publicado}{Improving delivery delay in social-based message forwarding in Delay Tolerant Networks. In: Proceedings of the 2016 workshop on Fostering Latin-American Research in Data Communication Networks, Florianópolis. LANCOMM 2016.}
\cvitem{Poster Publicado}{Autoarquivamento de Teses e Dissertações da UFAM: Prospecção do Processo. In: Conferência Luso-Brasileira sobre Acesso Aberto, Lisboa, Portugal. CONFOA 2015.}
\cvitem{Poster Publicado}{Rede Norte de Repositórios Institucionais: Trajetória da Experiência de sua Constituição. In: Conferência Luso-Brasileira sobre Acesso Aberto, Lisboa, Portugal. CONFOA 2015.}
\cvitem{Artigo Publicado}{A statistical learning reputation system for opportunistic networks. In: Wireless Days 2014 IFIP, Rio de Janeiro. WD 2014.}
\cvitem{Artigo Publicado}{Uma Nova Política De Gerência De Buffer Para Redes Tolerantes Ao Atraso e Desconexões Baseada Em Relações Sociais. In: XXXI Simpósio Brasileiro de Telecomunicações, 2013, Fortaleza. SBrT 2013.}

\section{Prêmios e Títulos}
\cventry{2013}{Menção honrosa - Etapa Nacional da Maratona de Programação 2013 (The 2013 ACM-ICPC South America Brazil)}{}{}{}{Sociedade Brasileira de Computação.}
\cventry{2013}{1º lugar - Etapa Regional da Maratona de Programação 2013 (The 2013 ACM-ICPC South America Brazil - First Phase)}{}{}{}{Sociedade Brasileira de Computação.}
\cventry{2013}{1º lugar - Maratona de Teste de Software}{}{}{}{Encontro Amazônico de Teste de Software.}
\cventry{2012}{Menção honrosa - Etapa Regional da Maratona de Programação (The 2012 ACM-ICPC South America Brazil - First Phase)}{}{}{}{Sociedade Brasileira de Computação.}

\section{Habilidades}

\cvitem{Básico}{\textsc{R}, \textsc{css}, \textsc{PHP}, \textsc{python}}
\cvitem{Intermediário}{\LaTeX, \textsc{html}, API Rest, \textsc{javascript}, \textsc{Java}, \textit{Network Monitoring} (Zabbix), CI/DC (Jenkins), Virtualização (\textsc{VMware})}
\cvitem{Avançado}{\textsc{Grails}, \textsc{Groovy}, \textsc{C/C++}, Linux Server, \textsc{bash}, containerização (Rancher, \textsc{Docker}), SQL, \textsc{Postgres}, \textsc{MySQL/MariaDB}}

%\cvitem{title}{ \emph{Title} }
%\cvitemwithcomment{Language 1}{Skill level}{Comment}
%\cvdoubleitem{category X}{XXX, YYY, ZZZ}{category Y}{XXX, YYY, ZZZ}
%\cvlistitem{Item 1}
%\cvlistdoubleitem{Item 2}{Item 3}
%%% ...

\bibliography{publications}
\end{document}
